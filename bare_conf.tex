\documentclass[monografia]{ppgccufscar}
%\hyphenation{op-tical net-works semi-conduc-tor}


%\usepackage[latin1]{inputenc}
%\usepackage[brazil]{babel}
%\usepackage[T1]{fontenc}

\usepackage[brazil]{babel}
\usepackage[utf8]{inputenc}
\usepackage[pdftex]{graphicx}
\usepackage{ifpdf}
\usepackage[alf]{abntcite}
\usepackage{subfig}
\usepackage{algorithm2e}
\usepackage{url}
\usepackage{enumerate}
\usepackage{indentfirst}
\setlength{\parindent}{0cm}


\titulo{Clusterização Routeflow usando MongoBD}
\autor{DAVI GONÇALES PETERLINI \\ FERNANDO CAPPI \\ LUIZ GUSTAVO MAION} 
\orientador[Orientador]{XXXXXXXXXXXXXXXXXX}
\coorientador[Coorientador]{Dr. Christian Esteve Rothenberg}
%\areaconcentracao{Sistemas Distribuídos e Redes}
\data{2013}


\begin{document} 
\capa
\folhaderosto
%\begin{folhaderosto}
%\end{folhaderosto}
\dataaprovacao{00}{Dezembro}{2013}
\begin{folhadeaprovacao}
\examinador{Prof. Dr. XXXXXXXXXXX }{USF - Itatiba}
\examinador{Prof. Dr. XXXXXXXXXXX }{USF - Itatiba}
\end{folhadeaprovacao}

\begin{resumo}
Nos últimos anos o protocolo \textit{OpenFlow} 
vem aumentando a visibilidade das tecnologias 
de redes definidas por software, fazendo com 
que um número cada vez maior de pesquisadores 
e desenvolvedores o adotem como principal
 ferramenta para simulações ou aplicações em
 ambientes reais. O projeto comunitário
 \textit{RouteFlow}, liderado pela Fundação 
\textit{CPqD}, propõe uma plataforma de 
roteamento IP definido por software (do
 termo em inglês, \textit{software-defined networking}) 
baseado no protocolo \textit{OpenFlow}, que permite 
uma separação efetiva do plano de controle do plano 
de encaminhamento dos equipamentos de rede. O 
fato do sistema ter sido criado com o código totalmente
 aberto fez com que o número de usuários aumenta-se 
consideravelmente incentivando a equipe de 
desenvolvedores a atualizá-la constantemente 
para agregar cada vez mais ferramentas e 
tecnologias. O \textit{RouteFlow} faz a 
manipulação do protocolo \textit{OpenFlow} 
utilizando os softwares de controle mais famosos 
da literatura, o \textit{NOX}, criado totalmente 
em \textit{C++} e o \textit{POX}, criado 
totalmente em \textit{Python}. Para aumentar a 
capacidade do \textit{RouteFlow} o trabalho em 
questão descreve a adição de suporte à um novo 
software de controle, o \textit{Floodlight}, sendo 
criado totalmente em \textit{Java}. Sendo assim 
o \textit{RouteFlow} ganhará suporte a mais uma
 tecnologia mantendo-se sempre na vanguarda
 das tecnologias de redes definidas por software.     

\palavraschave{Redes Definidas por Software}
\end{resumo}

\begin{abstract}

In the last years the \textit{OpenFlow} 
protocol has increased the visibility of the software
 defined network technologies, causing a growing 
number of researchers and developers to adopt it as 
the main tool for simulations or applications in real 
environments. The \textit{RouteFlow} community 
project, led by CPqD Foundation, proposes a platform
 defined by IP routing software based on 
the \textit{OpenFlow} protocol, which allows 
effective separation of the control plane of routing
 equipment plan network. The fact that the system
 has been created with fully open source has caused
 the number of users increases considerably encouraging
 the development team to update it constantly adding 
more and more tools and technologies. 
The \textit{RouteFlow} manipulate  the 
\textit{OpenFlow} protocol using the most famous 
controllers of the literature, \textit{NOX}, created 
entirely in \textit{C++} and \textit{POX}, created 
entirely in Python. To increase the capacity of 
the \textit{RouteFlow}, this work describes the 
addiction of the support for a new controller, 
\textit{Floodlight}, being created entirely in 
\textit{Java}. Thus the \textit{RouteFlow} win 
support more technology always staying at the
 forefront of the software defined network technologies.

\keywords{Software Defined Network}
\end{abstract}

\listoffigures
\listoftables

%% defina aqui o seu glossario
\acronym{ISP}{\textit{Internet Service Provider}}
\acronym{IP}{\textit{Internet Protocol}}
\acronym{UDP}{\textit{User Datagram Protocol}}
\acronym{TCP}{\textit{Transmission Control Protocol}}
\acronym{OF}{\textit{OpenFlow}}
\acronym{ARP}{\textit{Address Resolution Protocol}}
\acronym{MPLS}{\textit{Multiprotocol Label Switching}}
\acronym{IPC}{\textit{Inter-Process Communication}}
\acronym{REST}{\textit{Representational State Transfer}}
\acronym{OSPF}{\textit{Open Shortest Path First}}
\acronym{BGP}{\textit{Border Gateway Protocol}}
\acronym{RIP}{\textit{Routing Information Protocol}}
\acronym{JSON}{\textit{JavaScript Object Notation}}
\acronym{API}{\textit{Application Programming Interface}}
\acronym{SNMP}{\textit{Simple Network Management Protocol}}
\listofacronyms

%% sumario
\tableofcontents

\input{introducao}
\input{redes_definidas_por_software}
\input{arquitetura_routeflow}
\input{proxy_routeflow}
\input{resultados}
\input{conclusao}
\input{agradecimentos}
\bibliographystyle{abnt}
\nocite{*}
\bibliography{bare_conf}

\end{document}
